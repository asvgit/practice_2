\section{Задание на практику}
	В результате прохождения практики необходимо разработать проект для оформления
	научных работ согласно установленным стандартам ВУЗа\cite{norm}.

	В ходе практики должны быть освоены компетенции:
		\begin{itemize}
			\item способность совершенствовать и развивать свой интеллектуальный и общекультурный уровень;
			\item способность к самостоятельному обучению новым методам исследования, к изменению научного и научно-производственного профиля своей профессиональной деятельности;
			\item умение оформлять отчеты о проведенной научно-исследовательской работе и подготавливать публикации по результатам исследования.
		\end{itemize}

\newpage
\section{Введение}
	На протяжении всего учебного процесса, студентам приходится оформлять достаточно сложные документы,
	будь то лабораторная с не тривиальными вычислениями, курсовая работа (проект) со сложной структурой документа
	или отчеты по различным видам работ. 
	Более того, следует вести строгий контроль структуры и ссылок в дипломных и курсовых работах, в которых более десятка страниц, хоть и дело не объеме, а в том,
	как расставить, выровнять, расположить текст, изображения и формулы.

	Справится с выше описанной проблемой поможет любой текстовый редактор (тестовый процессор). И к программному обеспечению следует отнестись более качественно и профессионально,
	так как оно должно быть всегда под рукой и отвечать всем необходимым требованиям, то есть должно быть специализированным программным продуктом.

	Выбор подходящего программного обеспечения увеличивает эффективность научных исследований, тем самым экономит отведённые а исследование средства и время.
