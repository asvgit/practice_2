\stepcounter{mysection}\section{\arabic{mysection} Теоретическая часть}
	\subsection{Введение в \LaTeX}
		LaTeX -- наиболее популярный набор макрорасширений (или макропакет) системы компьютерной вёрстки TeX,
		который облегчает набор сложных документов. В типографском наборе системы TeX форматируется традиционно как \LaTeX .

		Общий внешний вид документа в LaTeX определяется стилевым файлом. Существует несколько стандартных стилевых файлов для статей,
		книг, писем и т. д., кроме того, многие издательства и журналы предоставляют свои собственные стилевые файлы,
		что позволяет быстро оформить публикацию, соответствующую стандартам издания. Всё это и выше превиденное в данной главе
		описано в статье свободной энциклопедии.

	\subsection{Сильные и слабые стороны \LaTeX}
		\subsubsection*{Преимущества}
			Основной плюс системы -- это возможность ввода формул различной сложности. Более того, это весьма просто и быстро.

			Вторым пунктом является гибкость системы, которая позволяет автоматизировать практически любое действие или процесс.
			Модульность позволяет специализировать систему под совершенно любую задачу. Более того, каждый модуль сопровождается хорошей документацией, что является ещё одним преимуществом. 

			\LaTeX относится к свободному программному обеспечению. Эта система была разработана американским профессором информатики, Дональдом Кнутом.

			И последним пунктом, хоть и не самым актуальным, является низкая потребность системы к ресурсам компьютера.

		\subsubsection*{Недостатки}
			\LaTeX весьма сложен в изучении и применении. И после достаточно запутанной установки, изучение команд и особенностей \LaTeX
			для начинающего пользователя является непростой задачей.

			Вторым пунктом является ее гибкость в настройке, хотя и можно дорабатывать и
			настраивать систему предостаточно, но иногда настройка выливается в непосильную задачу.

			Третий пункт заключается в том, что \LaTeX - не WYSIWIG система. Это означает, что пользователь при работе с исходным файлом не видит изменений на экране.
			А также требуется компиляция исходных файлов для просмотра результатов. Зачастую ожидания автора не оправдываются.

	\subsection{Применение \LaTeX}
		Зачастую \LaTeX используется в кругах научных деятелей, так как помогает представить оформление научных трудов для любой редакции,
		также позволяет с лёгкостью контролировать материал представляемый в документах. И учитывая то, что эту систему можно интегрировать с
		другими удобными пользователю системами, функционал инструментария для работы с текстом становится практически безграничным.
		
		Для редактирования исходных файлов можно использовать vim, emacs, visual code и множество других бесплатных редакторов, которые позваляют использовать огромное количество
		плагинов, существенно улучшая работу пользователя.
		
		Также можно использовать систему контроля версий и многое другое, чем не могут похвастаться офисные текстовые редакторы (процессоры).
